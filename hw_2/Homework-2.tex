\documentclass[]{article}
\usepackage{lmodern}
\usepackage{amssymb,amsmath}
\usepackage{ifxetex,ifluatex}
\usepackage{fixltx2e} % provides \textsubscript
\ifnum 0\ifxetex 1\fi\ifluatex 1\fi=0 % if pdftex
  \usepackage[T1]{fontenc}
  \usepackage[utf8]{inputenc}
\else % if luatex or xelatex
  \ifxetex
    \usepackage{mathspec}
  \else
    \usepackage{fontspec}
  \fi
  \defaultfontfeatures{Ligatures=TeX,Scale=MatchLowercase}
\fi
% use upquote if available, for straight quotes in verbatim environments
\IfFileExists{upquote.sty}{\usepackage{upquote}}{}
% use microtype if available
\IfFileExists{microtype.sty}{%
\usepackage[]{microtype}
\UseMicrotypeSet[protrusion]{basicmath} % disable protrusion for tt fonts
}{}
\PassOptionsToPackage{hyphens}{url} % url is loaded by hyperref
\usepackage[unicode=true]{hyperref}
\hypersetup{
            pdftitle={Homework 2},
            pdfauthor={Madeline Berger},
            pdfborder={0 0 0},
            breaklinks=true}
\urlstyle{same}  % don't use monospace font for urls
\usepackage[margin=1in]{geometry}
\usepackage{color}
\usepackage{fancyvrb}
\newcommand{\VerbBar}{|}
\newcommand{\VERB}{\Verb[commandchars=\\\{\}]}
\DefineVerbatimEnvironment{Highlighting}{Verbatim}{commandchars=\\\{\}}
% Add ',fontsize=\small' for more characters per line
\usepackage{framed}
\definecolor{shadecolor}{RGB}{248,248,248}
\newenvironment{Shaded}{\begin{snugshade}}{\end{snugshade}}
\newcommand{\KeywordTok}[1]{\textcolor[rgb]{0.13,0.29,0.53}{\textbf{#1}}}
\newcommand{\DataTypeTok}[1]{\textcolor[rgb]{0.13,0.29,0.53}{#1}}
\newcommand{\DecValTok}[1]{\textcolor[rgb]{0.00,0.00,0.81}{#1}}
\newcommand{\BaseNTok}[1]{\textcolor[rgb]{0.00,0.00,0.81}{#1}}
\newcommand{\FloatTok}[1]{\textcolor[rgb]{0.00,0.00,0.81}{#1}}
\newcommand{\ConstantTok}[1]{\textcolor[rgb]{0.00,0.00,0.00}{#1}}
\newcommand{\CharTok}[1]{\textcolor[rgb]{0.31,0.60,0.02}{#1}}
\newcommand{\SpecialCharTok}[1]{\textcolor[rgb]{0.00,0.00,0.00}{#1}}
\newcommand{\StringTok}[1]{\textcolor[rgb]{0.31,0.60,0.02}{#1}}
\newcommand{\VerbatimStringTok}[1]{\textcolor[rgb]{0.31,0.60,0.02}{#1}}
\newcommand{\SpecialStringTok}[1]{\textcolor[rgb]{0.31,0.60,0.02}{#1}}
\newcommand{\ImportTok}[1]{#1}
\newcommand{\CommentTok}[1]{\textcolor[rgb]{0.56,0.35,0.01}{\textit{#1}}}
\newcommand{\DocumentationTok}[1]{\textcolor[rgb]{0.56,0.35,0.01}{\textbf{\textit{#1}}}}
\newcommand{\AnnotationTok}[1]{\textcolor[rgb]{0.56,0.35,0.01}{\textbf{\textit{#1}}}}
\newcommand{\CommentVarTok}[1]{\textcolor[rgb]{0.56,0.35,0.01}{\textbf{\textit{#1}}}}
\newcommand{\OtherTok}[1]{\textcolor[rgb]{0.56,0.35,0.01}{#1}}
\newcommand{\FunctionTok}[1]{\textcolor[rgb]{0.00,0.00,0.00}{#1}}
\newcommand{\VariableTok}[1]{\textcolor[rgb]{0.00,0.00,0.00}{#1}}
\newcommand{\ControlFlowTok}[1]{\textcolor[rgb]{0.13,0.29,0.53}{\textbf{#1}}}
\newcommand{\OperatorTok}[1]{\textcolor[rgb]{0.81,0.36,0.00}{\textbf{#1}}}
\newcommand{\BuiltInTok}[1]{#1}
\newcommand{\ExtensionTok}[1]{#1}
\newcommand{\PreprocessorTok}[1]{\textcolor[rgb]{0.56,0.35,0.01}{\textit{#1}}}
\newcommand{\AttributeTok}[1]{\textcolor[rgb]{0.77,0.63,0.00}{#1}}
\newcommand{\RegionMarkerTok}[1]{#1}
\newcommand{\InformationTok}[1]{\textcolor[rgb]{0.56,0.35,0.01}{\textbf{\textit{#1}}}}
\newcommand{\WarningTok}[1]{\textcolor[rgb]{0.56,0.35,0.01}{\textbf{\textit{#1}}}}
\newcommand{\AlertTok}[1]{\textcolor[rgb]{0.94,0.16,0.16}{#1}}
\newcommand{\ErrorTok}[1]{\textcolor[rgb]{0.64,0.00,0.00}{\textbf{#1}}}
\newcommand{\NormalTok}[1]{#1}
\usepackage{graphicx,grffile}
\makeatletter
\def\maxwidth{\ifdim\Gin@nat@width>\linewidth\linewidth\else\Gin@nat@width\fi}
\def\maxheight{\ifdim\Gin@nat@height>\textheight\textheight\else\Gin@nat@height\fi}
\makeatother
% Scale images if necessary, so that they will not overflow the page
% margins by default, and it is still possible to overwrite the defaults
% using explicit options in \includegraphics[width, height, ...]{}
\setkeys{Gin}{width=\maxwidth,height=\maxheight,keepaspectratio}
\IfFileExists{parskip.sty}{%
\usepackage{parskip}
}{% else
\setlength{\parindent}{0pt}
\setlength{\parskip}{6pt plus 2pt minus 1pt}
}
\setlength{\emergencystretch}{3em}  % prevent overfull lines
\providecommand{\tightlist}{%
  \setlength{\itemsep}{0pt}\setlength{\parskip}{0pt}}
\setcounter{secnumdepth}{0}
% Redefines (sub)paragraphs to behave more like sections
\ifx\paragraph\undefined\else
\let\oldparagraph\paragraph
\renewcommand{\paragraph}[1]{\oldparagraph{#1}\mbox{}}
\fi
\ifx\subparagraph\undefined\else
\let\oldsubparagraph\subparagraph
\renewcommand{\subparagraph}[1]{\oldsubparagraph{#1}\mbox{}}
\fi

% set default figure placement to htbp
\makeatletter
\def\fps@figure{htbp}
\makeatother


\title{Homework 2}
\author{Madeline Berger}
\date{1/28/2020}

\begin{document}
\maketitle

\subsubsection{Analyzing growth model for invasive species with managed
hunting}\label{analyzing-growth-model-for-invasive-species-with-managed-hunting}

\(f(N) = \frac{dN}{dt} = rN(1-\frac{N}{K}) - P(\frac{aN}{1 + ahN})\)

\textbf{1. Make a graph of dN/dt vs.~N, for particular values of the
parameters} Use r = 0.05, K = 2000, P = 4, a = 0.01, and h = 0.2.

\begin{Shaded}
\begin{Highlighting}[]
\CommentTok{#define parameters}

\NormalTok{r <-}\StringTok{ }\FloatTok{0.05}
\NormalTok{K <-}\StringTok{ }\DecValTok{2000}
\NormalTok{P <-}\StringTok{ }\DecValTok{4}
\NormalTok{a <-}\StringTok{ }\FloatTok{0.05} \CommentTok{#this was changed}
\NormalTok{h <-}\StringTok{ }\FloatTok{0.2}

\CommentTok{#write out function}
\NormalTok{f =}\StringTok{ }\ControlFlowTok{function}\NormalTok{(N)\{}
\NormalTok{  r }\OperatorTok{*}\StringTok{ }\NormalTok{N }\OperatorTok{*}\StringTok{ }\NormalTok{(}\DecValTok{1} \OperatorTok{-}\StringTok{ }\NormalTok{(N}\OperatorTok{/}\NormalTok{K)) }\OperatorTok{-}\StringTok{ }\NormalTok{P }\OperatorTok{*}\StringTok{ }\NormalTok{((a}\OperatorTok{*}\NormalTok{N)}\OperatorTok{/}\NormalTok{(}\DecValTok{1} \OperatorTok{+}\StringTok{ }\NormalTok{a}\OperatorTok{*}\NormalTok{h}\OperatorTok{*}\NormalTok{N))}
\NormalTok{\}}

\NormalTok{growth_plot <-}\StringTok{ }\KeywordTok{ggplot}\NormalTok{(}\KeywordTok{data.frame}\NormalTok{(}\DataTypeTok{N =} \DecValTok{0}\OperatorTok{:}\DecValTok{2000}\NormalTok{), }\KeywordTok{aes}\NormalTok{(}\DataTypeTok{x =}\NormalTok{ N)) }\OperatorTok{+}\StringTok{ }
\StringTok{  }\KeywordTok{stat_function}\NormalTok{(}\DataTypeTok{fun =} \StringTok{"f"}\NormalTok{, }\DataTypeTok{color =} \StringTok{"red"}\NormalTok{) }\OperatorTok{+}
\StringTok{  }\KeywordTok{geom_hline}\NormalTok{(}\DataTypeTok{yintercept =} \DecValTok{0}\NormalTok{) }\OperatorTok{+}
\StringTok{  }\KeywordTok{ylab}\NormalTok{(}\StringTok{"dN/dt"}\NormalTok{) }\OperatorTok{+}\StringTok{ }
\StringTok{  }\KeywordTok{theme_minimal}\NormalTok{()}

\NormalTok{growth_plot}
\end{Highlighting}
\end{Shaded}

\includegraphics{Homework-2_files/figure-latex/unnamed-chunk-1-1.pdf}

\textbf{2.Based on this graph, how many equilibria are there? Which ones
are stable?}

Based on this graph there is one equilibria at zero, and it is unstable
because the slope at that point is positive. There is another equilibria
at N = 1500, which is stable as the growth rate is negative for N
\textgreater{} 1500 and positive for N \textless{} 1500.

\textbf{3. Returning to the full model with arbitrary parameter values,
is there an equilibrium at N = 0 for all plausible parameter values? For
what values of hunter number (P), expressed in terms of the other
parameters of the model, is the zero equilibrium locally stable? If your
goal is to eliminate the insvasive species, what does this tell you
about how many hunters you need?}

Looking at the expression, we see algebraically that when N = 0,
\(\frac{dN}{dt}\) will always be zero regardless of the other
parameters, as the first term is multiplied by zero, and the second term
has a zero in the denominator reducing the entire expression to zero.

To find the values of P that would make the population stable, the model
must first be differentiated in terms of N, which results in the
equation:

\(\frac{df(N)}{dN} = rN(1-\frac{2N}{K}) - P(\frac{a}{(1 + ahN)^2})\)

Since we are evaluating this expression at N = 0, the equation becomes:

\(\frac{df(N)}{dN} = r-P*a\)

The equilibrium is stable when this expression is a negative value. Or
when:

\(0 = r-P*a\) solving for P:

\(P = \frac{r}{a}\)

This implies that the number of hunters needed depends both on the
growth rate of the invasive species as well as the ``per pray attack
rate'', which is analagous to the number of shots a hunter takes (??).
For example, if the growth rate increases but the hunters do not
increase their shots, then more hunters are needed. This might apply to
an animal that is difficult to hunt, which wild boars most likely are.

\emph{attack rate = a rate at which the hunter encounters and
successfully kills}

\begin{Shaded}
\begin{Highlighting}[]
\CommentTok{#what is the derivative of the model equation? gives you the slope (just to double check my calculations)}
\NormalTok{dFN <-}\StringTok{ }\KeywordTok{Deriv}\NormalTok{(f)}

\KeywordTok{Simplify}\NormalTok{(dFN)}
\end{Highlighting}
\end{Shaded}

\begin{verbatim}
## function (N) 
## {
##     .e1 <- 1 + a * h * N
##     N * (a^2 * h * P/.e1^2 - r/K) + r * (1 - N/K) - a * P/.e1
## }
\end{verbatim}

\textbf{4. Write down the equation you would need to solve in order to
find the value of any non-zero equilibria. If you enjoy doing algebra,
you may use the quadratic formula to find the values of N that satisfy
this equation, but that is entirely optional (the result is a rather
complicated expression!).}

To find any non-zero equilibria, set the intial equation to zero, and
solve for N:

\(0 = r*N*(1-\frac{N}{k})-P*(\frac{aN}{1 + ahN})\)

\textbf{5. Because the algebra is tedious, and the result is so complex
that it doesn't give a lot of insight, it's useful to do some more
graphical analysis. In particular, we break the equation into its two
component parts, the intrinsic population growth}
\(r*N*(1-\frac{N}{k})\) \textbf{and hunting} \(P*(\frac{aN}{1 + ahN})\)

\begin{enumerate}
\def\labelenumi{\alph{enumi}.}
\tightlist
\item
  Using the same parameters as before, graph both the intrinsic growth
  rate and hunting rate as functions of N on the same graph.
\end{enumerate}

\begin{Shaded}
\begin{Highlighting}[]
\NormalTok{f_growth =}\StringTok{ }\ControlFlowTok{function}\NormalTok{(N) \{}
\NormalTok{  r }\OperatorTok{*}\StringTok{ }\NormalTok{N }\OperatorTok{*}\StringTok{ }\NormalTok{(}\DecValTok{1} \OperatorTok{-}\NormalTok{(N}\OperatorTok{/}\NormalTok{K))}
\NormalTok{  \}}

\NormalTok{f_hunt =}\StringTok{ }\ControlFlowTok{function}\NormalTok{(N, }\DataTypeTok{P =} \DecValTok{4}\NormalTok{)\{}
\NormalTok{  P }\OperatorTok{*}\StringTok{ }\NormalTok{(a}\OperatorTok{*}\NormalTok{N}\OperatorTok{/}\NormalTok{(}\DecValTok{1} \OperatorTok{+}\StringTok{ }\NormalTok{a}\OperatorTok{*}\NormalTok{h}\OperatorTok{*}\NormalTok{N))}
\NormalTok{\}}

\NormalTok{separate_plot <-}\StringTok{ }\KeywordTok{ggplot}\NormalTok{(}\KeywordTok{data.frame}\NormalTok{(}\DataTypeTok{N =} \DecValTok{0}\OperatorTok{:}\DecValTok{2000}\NormalTok{), }\KeywordTok{aes}\NormalTok{(}\DataTypeTok{x =}\NormalTok{ N))}\OperatorTok{+}\StringTok{ }
\StringTok{  }\KeywordTok{stat_function}\NormalTok{(}\DataTypeTok{fun =}\NormalTok{ f_growth, }\DataTypeTok{color =} \StringTok{"red"}\NormalTok{) }\OperatorTok{+}
\StringTok{  }\KeywordTok{stat_function}\NormalTok{(}\DataTypeTok{fun =}\NormalTok{ f_hunt, }\DataTypeTok{linetype =} \StringTok{"dotted"}\NormalTok{)}\OperatorTok{+}
\StringTok{  }\KeywordTok{geom_hline}\NormalTok{(}\DataTypeTok{yintercept =} \DecValTok{0}\NormalTok{) }\OperatorTok{+}
\StringTok{  }\KeywordTok{ylab}\NormalTok{(}\StringTok{"dN/dt"}\NormalTok{)}\OperatorTok{+}
\StringTok{  }\KeywordTok{theme_minimal}\NormalTok{()}

  
\NormalTok{separate_plot}
\end{Highlighting}
\end{Shaded}

\includegraphics{Homework-2_files/figure-latex/unnamed-chunk-3-1.pdf}

\textbf{b. What do you expect will happen to the population when the
hunting rate is greater than the intrinsic growth rate? When it is less?
When they are equal?}

When the hunting rate exceeds the intrinsic growth rate, we would expect
the population to decrease. When it is less, the population would be
expected to increase. At the point of intersection the population is in
equilibrium because birth rate (dn/dt) is equal to mortality (hunting
rate)

\textbf{c. How do the patterns you see on this graph relate to the ones
in problem 1?}

This matches what we see in number one in that

\textbf{6. Now make two similar graphs, keeping all the parameters the
same but setting P = 1 in one graph and P = 6 in the other. How many
equilbria are there in each case, and which are stable?}

\begin{Shaded}
\begin{Highlighting}[]
\NormalTok{separate_plot <-}\StringTok{ }\KeywordTok{ggplot}\NormalTok{(}\KeywordTok{data.frame}\NormalTok{(}\DataTypeTok{N =} \DecValTok{0}\OperatorTok{:}\DecValTok{2000}\NormalTok{), }\KeywordTok{aes}\NormalTok{(}\DataTypeTok{x =}\NormalTok{ N))}\OperatorTok{+}\StringTok{ }
\StringTok{  }\KeywordTok{stat_function}\NormalTok{(}\DataTypeTok{fun =}\NormalTok{ f_growth, }\DataTypeTok{color =} \StringTok{"red"}\NormalTok{) }\OperatorTok{+}
\StringTok{  }\KeywordTok{stat_function}\NormalTok{(}\DataTypeTok{fun =}\NormalTok{ f_hunt, }\DataTypeTok{args =} \KeywordTok{list}\NormalTok{(}\DataTypeTok{P =} \DecValTok{1}\NormalTok{), }\DataTypeTok{color =} \StringTok{"gray"}\NormalTok{, }\DataTypeTok{linetype =} \StringTok{"dashed"}\NormalTok{)}\OperatorTok{+}
\StringTok{  }\KeywordTok{stat_function}\NormalTok{(}\DataTypeTok{fun =}\NormalTok{ f_hunt, }\DataTypeTok{args =} \KeywordTok{list}\NormalTok{(}\DataTypeTok{P =} \DecValTok{6}\NormalTok{), }\DataTypeTok{linetype =} \StringTok{"dotted"}\NormalTok{)}\OperatorTok{+}
\StringTok{  }\KeywordTok{geom_hline}\NormalTok{(}\DataTypeTok{yintercept =} \DecValTok{0}\NormalTok{) }\OperatorTok{+}
\StringTok{  }\KeywordTok{ylab}\NormalTok{(}\StringTok{"dN/dt"}\NormalTok{)}\OperatorTok{+}
\StringTok{  }\KeywordTok{theme_minimal}\NormalTok{()}

\NormalTok{separate_plot}
\end{Highlighting}
\end{Shaded}

\includegraphics{Homework-2_files/figure-latex/unnamed-chunk-4-1.pdf}

When P = 1, or when there is only one hunter, there are three
equilibria: at N = 0, N = K, and when the population growth rate is
equal to the hunting rate. Stable equilibria occur at N = K, and at the
intersection point of the two lines

When P = 6 (six hunters), there is only one equilibrium at N = 0. This
makes sense since the hunting rate, which can be though as mortality,
always exceeds the birth rate (dN/dt), keeping the population from being
able to increase to any stable point beyond zero.

\textbf{7. The situation in problem 5 is an example of bistability, like
in the strong Allee effect. It has important management implications.}

\textbf{a. What is the domain of attraction of the zero equlibrium
(approximately---you can estimate it from the graph) b, What is the
domain of attraction of the largest equibrium?}

Looking back at the first graph created, from N = 0 to appx N = 400 the
population will be attracted to the zero equilibrium since between those
values, the growth rate is negative. From 400 to infitinty it will be
attracted to N = 1500, a stable equilibrium. This is because the growth
rate is positive for any value between N = 400 and N = 1500, causing the
population to approach 1500, whereas after N = 1500 the growth rate is
negative, causing the population to decrease back towards 1500.

\textbf{b. If you noticed the arrival of the species soon after it
arrived, and initiated control activities when it had reached N = 100
individuals, would you be able to extirpate it with 4 hunters?}

Yes - From graph 1, when P = 4, N = 100 falls within the domain of
attraction of the zero equilibrium. Therefore, even if the population
reaches 100 somehow, the growth rate remains negative and the hunters
will be able to eliminate the species.

\textbf{c. What about if the population was already at carrying capacity
when you initiated control activities?}

If the species reaches carrying capacity before any management begins,
introducing 4 hunters will help to quell the population slighly by
reducing it to N = 1500. However, the population will remain at 1500 as
long as the birth rate remains the same and the hunting efficiency
(``a'') remains the same as well. This is because populations of
invasive species greater than N = 400 occur within the domain of
attraction for an equilibrium of 1500, not zero.

\textbf{8. It can be instructive to see how the equilibrium values
depend on the number of hunters. Here you will make plots of this, using
the same parameter values as above. You already know the formula for the
zero equilibrium, N∗0; the formulas for the other two are:}

Use same parameters

\begin{Shaded}
\begin{Highlighting}[]
\NormalTok{d <-}\StringTok{ }\DecValTok{1}\OperatorTok{/}\NormalTok{a}\OperatorTok{*}\NormalTok{h}

\NormalTok{f_eq_}\DecValTok{1}\NormalTok{ <-}\StringTok{ }\ControlFlowTok{function}\NormalTok{(P)\{}
\NormalTok{  .}\DecValTok{5} \OperatorTok{*}\StringTok{ }\NormalTok{((K }\OperatorTok{-}\StringTok{ }\NormalTok{d) }\OperatorTok{-}\StringTok{ }\KeywordTok{sqrt}\NormalTok{((K}\OperatorTok{-}\NormalTok{d)}\OperatorTok{^}\DecValTok{2} \OperatorTok{+}\StringTok{ }\NormalTok{(}\DecValTok{4}\OperatorTok{*}\NormalTok{K}\OperatorTok{*}\NormalTok{d)}\OperatorTok{/}\NormalTok{r}\OperatorTok{*}\NormalTok{(r }\OperatorTok{-}\StringTok{ }\NormalTok{a}\OperatorTok{*}\NormalTok{P)))}
\NormalTok{\}}

\NormalTok{f_eq_}\DecValTok{2}\NormalTok{ <-}\StringTok{ }\ControlFlowTok{function}\NormalTok{(P)\{}
\NormalTok{  .}\DecValTok{5} \OperatorTok{*}\StringTok{ }\NormalTok{((K }\OperatorTok{-}\StringTok{ }\NormalTok{d) }\OperatorTok{+}\StringTok{ }\KeywordTok{sqrt}\NormalTok{((K}\OperatorTok{-}\NormalTok{d)}\OperatorTok{^}\DecValTok{2} \OperatorTok{+}\StringTok{ }\NormalTok{(}\DecValTok{4}\OperatorTok{*}\NormalTok{K}\OperatorTok{*}\NormalTok{d)}\OperatorTok{/}\NormalTok{r}\OperatorTok{*}\NormalTok{(r }\OperatorTok{-}\StringTok{ }\NormalTok{a}\OperatorTok{*}\NormalTok{P)))}
\NormalTok{\}}


\NormalTok{eq_plot<-}\StringTok{ }\KeywordTok{ggplot}\NormalTok{(}\KeywordTok{data.frame}\NormalTok{(}\DataTypeTok{P =} \DecValTok{0}\OperatorTok{:}\DecValTok{7}\NormalTok{), }\KeywordTok{aes}\NormalTok{(}\DataTypeTok{x =}\NormalTok{ P))}\OperatorTok{+}\StringTok{ }
\StringTok{  }\KeywordTok{stat_function}\NormalTok{(}\DataTypeTok{fun =}\NormalTok{ f_eq_}\DecValTok{1}\NormalTok{, }\DataTypeTok{color =} \StringTok{"red"}\NormalTok{) }\OperatorTok{+}
\StringTok{  }\KeywordTok{stat_function}\NormalTok{(}\DataTypeTok{fun =}\NormalTok{ f_eq_}\DecValTok{2}\NormalTok{, }\DataTypeTok{color =} \StringTok{"green"}\NormalTok{)}\OperatorTok{+}
\StringTok{  }\KeywordTok{geom_hline}\NormalTok{(}\DataTypeTok{yintercept =} \DecValTok{0}\NormalTok{) }\OperatorTok{+}
\StringTok{  }\KeywordTok{ylab}\NormalTok{(}\StringTok{"Equilibrium"}\NormalTok{) }\OperatorTok{+}\StringTok{ }
\StringTok{  }\KeywordTok{theme_minimal}\NormalTok{()}

\NormalTok{eq_plot}
\end{Highlighting}
\end{Shaded}

\includegraphics{Homework-2_files/figure-latex/unnamed-chunk-5-1.pdf}

\end{document}
