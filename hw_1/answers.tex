\documentclass[]{article}
\usepackage{lmodern}
\usepackage{amssymb,amsmath}
\usepackage{ifxetex,ifluatex}
\usepackage{fixltx2e} % provides \textsubscript
\ifnum 0\ifxetex 1\fi\ifluatex 1\fi=0 % if pdftex
  \usepackage[T1]{fontenc}
  \usepackage[utf8]{inputenc}
\else % if luatex or xelatex
  \ifxetex
    \usepackage{mathspec}
  \else
    \usepackage{fontspec}
  \fi
  \defaultfontfeatures{Ligatures=TeX,Scale=MatchLowercase}
\fi
% use upquote if available, for straight quotes in verbatim environments
\IfFileExists{upquote.sty}{\usepackage{upquote}}{}
% use microtype if available
\IfFileExists{microtype.sty}{%
\usepackage[]{microtype}
\UseMicrotypeSet[protrusion]{basicmath} % disable protrusion for tt fonts
}{}
\PassOptionsToPackage{hyphens}{url} % url is loaded by hyperref
\usepackage[unicode=true]{hyperref}
\hypersetup{
            pdftitle={Homework 1},
            pdfauthor={Madeline Berger},
            pdfborder={0 0 0},
            breaklinks=true}
\urlstyle{same}  % don't use monospace font for urls
\usepackage[margin=1in]{geometry}
\usepackage{color}
\usepackage{fancyvrb}
\newcommand{\VerbBar}{|}
\newcommand{\VERB}{\Verb[commandchars=\\\{\}]}
\DefineVerbatimEnvironment{Highlighting}{Verbatim}{commandchars=\\\{\}}
% Add ',fontsize=\small' for more characters per line
\usepackage{framed}
\definecolor{shadecolor}{RGB}{248,248,248}
\newenvironment{Shaded}{\begin{snugshade}}{\end{snugshade}}
\newcommand{\KeywordTok}[1]{\textcolor[rgb]{0.13,0.29,0.53}{\textbf{#1}}}
\newcommand{\DataTypeTok}[1]{\textcolor[rgb]{0.13,0.29,0.53}{#1}}
\newcommand{\DecValTok}[1]{\textcolor[rgb]{0.00,0.00,0.81}{#1}}
\newcommand{\BaseNTok}[1]{\textcolor[rgb]{0.00,0.00,0.81}{#1}}
\newcommand{\FloatTok}[1]{\textcolor[rgb]{0.00,0.00,0.81}{#1}}
\newcommand{\ConstantTok}[1]{\textcolor[rgb]{0.00,0.00,0.00}{#1}}
\newcommand{\CharTok}[1]{\textcolor[rgb]{0.31,0.60,0.02}{#1}}
\newcommand{\SpecialCharTok}[1]{\textcolor[rgb]{0.00,0.00,0.00}{#1}}
\newcommand{\StringTok}[1]{\textcolor[rgb]{0.31,0.60,0.02}{#1}}
\newcommand{\VerbatimStringTok}[1]{\textcolor[rgb]{0.31,0.60,0.02}{#1}}
\newcommand{\SpecialStringTok}[1]{\textcolor[rgb]{0.31,0.60,0.02}{#1}}
\newcommand{\ImportTok}[1]{#1}
\newcommand{\CommentTok}[1]{\textcolor[rgb]{0.56,0.35,0.01}{\textit{#1}}}
\newcommand{\DocumentationTok}[1]{\textcolor[rgb]{0.56,0.35,0.01}{\textbf{\textit{#1}}}}
\newcommand{\AnnotationTok}[1]{\textcolor[rgb]{0.56,0.35,0.01}{\textbf{\textit{#1}}}}
\newcommand{\CommentVarTok}[1]{\textcolor[rgb]{0.56,0.35,0.01}{\textbf{\textit{#1}}}}
\newcommand{\OtherTok}[1]{\textcolor[rgb]{0.56,0.35,0.01}{#1}}
\newcommand{\FunctionTok}[1]{\textcolor[rgb]{0.00,0.00,0.00}{#1}}
\newcommand{\VariableTok}[1]{\textcolor[rgb]{0.00,0.00,0.00}{#1}}
\newcommand{\ControlFlowTok}[1]{\textcolor[rgb]{0.13,0.29,0.53}{\textbf{#1}}}
\newcommand{\OperatorTok}[1]{\textcolor[rgb]{0.81,0.36,0.00}{\textbf{#1}}}
\newcommand{\BuiltInTok}[1]{#1}
\newcommand{\ExtensionTok}[1]{#1}
\newcommand{\PreprocessorTok}[1]{\textcolor[rgb]{0.56,0.35,0.01}{\textit{#1}}}
\newcommand{\AttributeTok}[1]{\textcolor[rgb]{0.77,0.63,0.00}{#1}}
\newcommand{\RegionMarkerTok}[1]{#1}
\newcommand{\InformationTok}[1]{\textcolor[rgb]{0.56,0.35,0.01}{\textbf{\textit{#1}}}}
\newcommand{\WarningTok}[1]{\textcolor[rgb]{0.56,0.35,0.01}{\textbf{\textit{#1}}}}
\newcommand{\AlertTok}[1]{\textcolor[rgb]{0.94,0.16,0.16}{#1}}
\newcommand{\ErrorTok}[1]{\textcolor[rgb]{0.64,0.00,0.00}{\textbf{#1}}}
\newcommand{\NormalTok}[1]{#1}
\usepackage{graphicx,grffile}
\makeatletter
\def\maxwidth{\ifdim\Gin@nat@width>\linewidth\linewidth\else\Gin@nat@width\fi}
\def\maxheight{\ifdim\Gin@nat@height>\textheight\textheight\else\Gin@nat@height\fi}
\makeatother
% Scale images if necessary, so that they will not overflow the page
% margins by default, and it is still possible to overwrite the defaults
% using explicit options in \includegraphics[width, height, ...]{}
\setkeys{Gin}{width=\maxwidth,height=\maxheight,keepaspectratio}
\IfFileExists{parskip.sty}{%
\usepackage{parskip}
}{% else
\setlength{\parindent}{0pt}
\setlength{\parskip}{6pt plus 2pt minus 1pt}
}
\setlength{\emergencystretch}{3em}  % prevent overfull lines
\providecommand{\tightlist}{%
  \setlength{\itemsep}{0pt}\setlength{\parskip}{0pt}}
\setcounter{secnumdepth}{0}
% Redefines (sub)paragraphs to behave more like sections
\ifx\paragraph\undefined\else
\let\oldparagraph\paragraph
\renewcommand{\paragraph}[1]{\oldparagraph{#1}\mbox{}}
\fi
\ifx\subparagraph\undefined\else
\let\oldsubparagraph\subparagraph
\renewcommand{\subparagraph}[1]{\oldsubparagraph{#1}\mbox{}}
\fi

% set default figure placement to htbp
\makeatletter
\def\fps@figure{htbp}
\makeatother


\title{Homework 1}
\author{Madeline Berger}
\date{1/13/2020}

\begin{document}
\maketitle

\subsubsection{Part 1}\label{part-1}

\textbf{1. Give three reasons why a plant or animal might be patchily
distributed like this. For each, describe whether you would expect the
location of the high-density patches to be consistent from year to year
(1 point).}

\begin{enumerate}
\def\labelenumi{\arabic{enumi}.}
\item
  The landscape might be patched. Even at a small scales, landscapes may
  have distinct patches of vegetation or land cover that would cause
  populations to cluster in this pattern. An example might be a golf
  course, where there might exist small patches of unmanaged land or
  ponds between the managed greens. In this case, unless the land use of
  the area was altered, you would expect the patches to be consistent
  year to year. Another, more natural example could include a tidepool
  landscape.
\item
  Access to light, or other limiting abiotic environmental factors may
  be unevenly distributed across the landscape. An example might be a
  forest with some very tall canopy trees that block lights in large
  patches, causing shorter species to congregate in specific areas. This
  may not be totally consistent year to year depending on the growth
  rate of the canopy species, as leaf cover and branches may shift as
  the species grows.
\item
  Feeding behavior might also explain why certain species are
  distributed this way. Certain species of insects, fungi or plants may
  act as decomposers, attracting to areas where another animal has died.
  In this example, the high-density patches would not most likely not be
  consistent year to year.
\end{enumerate}

\subsubsection{Part 2 - Eureka Dune
grass}\label{part-2---eureka-dune-grass}

\textbf{2. Read in the data, and use a two-sample t-test (e.g., using
t.test() in R) to assess whether and how mean abundance changed from
2009 to 2010. Choose a level for α, and justify your choice. On the
basis of this analysis, what do you conclude about the change in the
grass's abundance? (1 point)}

\begin{Shaded}
\begin{Highlighting}[]
\CommentTok{#explore the means}
\NormalTok{mean_}\DecValTok{2009}\NormalTok{ <-}\StringTok{ }\KeywordTok{mean}\NormalTok{(swallenia}\OperatorTok{$}\NormalTok{count_}\DecValTok{2009}\NormalTok{)}

\NormalTok{mean_}\DecValTok{2010}\NormalTok{ <-}\StringTok{ }\KeywordTok{mean}\NormalTok{(swallenia}\OperatorTok{$}\NormalTok{count_}\DecValTok{2010}\NormalTok{)}

\CommentTok{#run the t.test}
\KeywordTok{t.test}\NormalTok{(swallenia}\OperatorTok{$}\NormalTok{count_}\DecValTok{2009}\NormalTok{, swallenia}\OperatorTok{$}\NormalTok{count_}\DecValTok{2010}\NormalTok{, }\DataTypeTok{conf.level =} \FloatTok{0.95}\NormalTok{)}
\end{Highlighting}
\end{Shaded}

Using a confidence level of 95\% (alpha = 0.05) it appears that there is
not a significant difference between the mean abundance of swallenia
between 2009 and 2010 (t\{10\} = -0.81791, p \textgreater{} 0.05). This
confidence level was used to minimize the risk of comitting a type 1
error, ie finding that there was a significant change in grass
population when there actually was not. The greater management
implications of a type 1 error in this case could be park managers
interpreting a significant increase in population as a sign that current
management is sufficient. This may overlook critical issues and prevent
managers and scientists to do a more in depth analysis of the area and
each site.

\textbf{3. Because the plant was counted in the same plots each year,
you can also do a paired t-test (R hint: use paired = TRUE in t.test()).
On the basis of this analysis, what do you conclude about the change in
the grass's abundance? (1 point)}

\begin{Shaded}
\begin{Highlighting}[]
\CommentTok{#run paired t.test}
\KeywordTok{t.test}\NormalTok{(swallenia}\OperatorTok{$}\NormalTok{count_}\DecValTok{2009}\NormalTok{, swallenia}\OperatorTok{$}\NormalTok{count_}\DecValTok{2010}\NormalTok{, }\DataTypeTok{paired =} \OtherTok{TRUE}\NormalTok{, }\DataTypeTok{conf.level =} \FloatTok{0.95}\NormalTok{)}
\end{Highlighting}
\end{Shaded}

On the basis of this analysis, there is sufficient evidence that there
is a difference in abunance of swallenia between the years 2009 and 2010
(t\{10\} = -2.41, p \textless{} 0.05).

\textbf{4. Which of these analyses is more appropriate? Why? (Hint:
think about your answers to question 1, and whether any apply here) (1
point)}

I believe the second (paired t.test) analysis is more appropriate as it
allows you to assess changes in abundace without confounds from any
variance across sites. For example, there might be slightly different
conditions at one cluster than another, therefore when comparing across
timescales it is more informative to understand the change in growth
given that the site and conditions are consistent.

\textbf{5. Write a short paragraph to the park superintendent describing
your finding about the changes (if any) to the dune grass population. (1
point)}

first test showed nothing, second test showed stuff, so suggests that
there are some sites that are doing much better than others and there
merits analysis of those individual sites.

\subsection{Part 3 - Grizzly Bears}\label{part-3---grizzly-bears}

\textbf{6. Open the file, and look at the data from 1959 to 1968. Did
the grizzly bear population decline over this period? Support your
conclusion with graphs, statistics, and logical reasoning. (2 points)}

\begin{Shaded}
\begin{Highlighting}[]
\CommentTok{#graph }
\NormalTok{grizzly_early <-}\StringTok{ }\NormalTok{grizzly }\OperatorTok\StringTok{ }
\StringTok{  }\KeywordTok{filter}\NormalTok{(Year }\OperatorTok{<}\StringTok{ }\DecValTok{1969}\NormalTok{)}

\NormalTok{grizzly_plot_}\DecValTok{1}\NormalTok{ <-}\StringTok{ }\KeywordTok{ggplot}\NormalTok{(grizzly_early, }\KeywordTok{aes}\NormalTok{(}\DataTypeTok{x =}\NormalTok{ Year, }\DataTypeTok{y =}\NormalTok{ N)) }\OperatorTok{+}
\StringTok{  }\KeywordTok{geom_line}\NormalTok{()}\OperatorTok{+}
\StringTok{  }\KeywordTok{theme_minimal}\NormalTok{()}\OperatorTok{+}
\StringTok{  }\KeywordTok{labs}\NormalTok{(}\DataTypeTok{x =} \StringTok{"Year"}\NormalTok{, }\DataTypeTok{y =} \StringTok{"Bear Abundance"}\NormalTok{, }\DataTypeTok{title =} \StringTok{"Yearly Grizzly Bear population 1959 - 1968"}\NormalTok{)}\OperatorTok{+}
\StringTok{  }\KeywordTok{theme}\NormalTok{(}\DataTypeTok{text =} \KeywordTok{element_text}\NormalTok{(}\DataTypeTok{family =} \StringTok{"Times"}\NormalTok{))}

\NormalTok{grizzly_plot_}\DecValTok{1}
\end{Highlighting}
\end{Shaded}

\includegraphics{answers_files/figure-latex/unnamed-chunk-3-1.pdf}

\begin{Shaded}
\begin{Highlighting}[]
\CommentTok{#statistics - how has the population declined?}

\NormalTok{early_diff <-}\StringTok{ }\NormalTok{(grizzly_early}\OperatorTok{$}\NormalTok{N[}\DecValTok{10}\NormalTok{]}\OperatorTok{-}\NormalTok{grizzly_early}\OperatorTok{$}\NormalTok{N[}\DecValTok{1}\NormalTok{])}

\NormalTok{early_pyr <-}\StringTok{ }\NormalTok{early_diff}\OperatorTok{/}\DecValTok{10}

\CommentTok{#the difference is -1/2 a bear per year}

\NormalTok{early_perc <-}\StringTok{ }\NormalTok{early_diff}\OperatorTok{/}\KeywordTok{sum}\NormalTok{(grizzly_early}\OperatorTok{$}\NormalTok{N)}

\CommentTok{#total decline of about 1%}

\NormalTok{early_model <-}\StringTok{ }\KeywordTok{lm}\NormalTok{(grizzly_early}\OperatorTok{$}\NormalTok{N }\OperatorTok{~}\StringTok{ }\NormalTok{grizzly_early}\OperatorTok{$}\NormalTok{Year)}

\KeywordTok{summary}\NormalTok{(early_model)}
\end{Highlighting}
\end{Shaded}

The graph suggests that the grizzly population did decline from 1959 to
1968. Logically, this would make sense as a main food source (the dumps)
was removed around this time, causing more human-bear conflict that can
lead to bear deaths. Statistically, we find that there are 5 fewer bears
reported in 1968 than in 1959 representing a total decline of about 1\%
during that time period.

\textbf{7. Select the data from 1969 to 1978. Did the population
continue to decline? Was the decline faster than the period prior to the
dump closures? (1 point)}

\begin{Shaded}
\begin{Highlighting}[]
\CommentTok{#plot}
\NormalTok{grizzly_mid <-}\StringTok{ }\NormalTok{grizzly }\OperatorTok\StringTok{ }
\StringTok{  }\KeywordTok{filter}\NormalTok{( Year }\OperatorTok{>}\StringTok{ }\DecValTok{1968} \OperatorTok{&}\StringTok{ }\NormalTok{Year }\OperatorTok{<}\StringTok{ }\DecValTok{1979}\NormalTok{)}

\NormalTok{grizzly_plot_}\DecValTok{2}\NormalTok{ <-}\StringTok{ }\KeywordTok{ggplot}\NormalTok{(grizzly_mid, }\KeywordTok{aes}\NormalTok{(}\DataTypeTok{x =}\NormalTok{ Year, }\DataTypeTok{y =}\NormalTok{ N)) }\OperatorTok{+}
\StringTok{  }\KeywordTok{geom_line}\NormalTok{()}\OperatorTok{+}
\StringTok{  }\KeywordTok{theme_minimal}\NormalTok{()}\OperatorTok{+}
\StringTok{  }\KeywordTok{labs}\NormalTok{(}\DataTypeTok{x =} \StringTok{"Year"}\NormalTok{, }\DataTypeTok{y =} \StringTok{"Bear Abundance"}\NormalTok{, }\DataTypeTok{title =} \StringTok{"Yearly Grizzly Bear population 1969 - 1978"}\NormalTok{)}\OperatorTok{+}
\StringTok{  }\KeywordTok{theme}\NormalTok{(}\DataTypeTok{text =} \KeywordTok{element_text}\NormalTok{(}\DataTypeTok{family =} \StringTok{"Times"}\NormalTok{))}

\NormalTok{grizzly_plot_}\DecValTok{2}
\end{Highlighting}
\end{Shaded}

\includegraphics{answers_files/figure-latex/unnamed-chunk-4-1.pdf}

\begin{Shaded}
\begin{Highlighting}[]
\CommentTok{#statistics - how has the population declined?}

\NormalTok{mid_diff <-}\StringTok{ }\NormalTok{(grizzly_mid}\OperatorTok{$}\NormalTok{N[}\DecValTok{10}\NormalTok{]}\OperatorTok{-}\NormalTok{grizzly_mid}\OperatorTok{$}\NormalTok{N[}\DecValTok{1}\NormalTok{])}

\NormalTok{mid_pyr <-}\StringTok{ }\NormalTok{mid_diff}\OperatorTok{/}\DecValTok{10}

\CommentTok{#the total percent decline}

\NormalTok{mid_perc <-}\StringTok{ }\NormalTok{mid_diff}\OperatorTok{/}\KeywordTok{sum}\NormalTok{(grizzly_early}\OperatorTok{$}\NormalTok{N)}

\CommentTok{#use linear regression to compare slopes - is the slope the rate of decline?}


\NormalTok{mid_model <-}\KeywordTok{lm}\NormalTok{(grizzly_mid}\OperatorTok{$}\NormalTok{N }\OperatorTok{~}\StringTok{ }\NormalTok{grizzly_mid}\OperatorTok{$}\NormalTok{Year)}

\KeywordTok{summary}\NormalTok{(mid_model)}
\end{Highlighting}
\end{Shaded}

The grizzly population continued to decline during this period. Between
1969 and 1978 the population decreased by 8 bears, representing close to
a 2\% decline overall. The rate of decline was similar to the previous
10 years. Using a simple linear regression we find that from 1959 to
1968, the average rate of decline was -0.763 bears per year, while from
1969 to 1978 the population declined on average -0.757 bears per year.

\textbf{8. Finally look at the data after 1978. Did population size
continue to change? What was the direction and magnitude of the trend?
(1 point)}

\begin{Shaded}
\begin{Highlighting}[]
\NormalTok{grizzly_late <-}\StringTok{ }\NormalTok{grizzly }\OperatorTok\StringTok{ }
\StringTok{  }\KeywordTok{filter}\NormalTok{( Year }\OperatorTok{>}\StringTok{ }\DecValTok{1978}\NormalTok{)}

\NormalTok{grizzly_plot_}\DecValTok{3}\NormalTok{ <-}\StringTok{ }\KeywordTok{ggplot}\NormalTok{(grizzly_late, }\KeywordTok{aes}\NormalTok{(}\DataTypeTok{x =}\NormalTok{ Year, }\DataTypeTok{y =}\NormalTok{ N)) }\OperatorTok{+}
\StringTok{  }\KeywordTok{geom_line}\NormalTok{()}\OperatorTok{+}
\StringTok{  }\KeywordTok{theme_minimal}\NormalTok{()}\OperatorTok{+}
\StringTok{  }\KeywordTok{labs}\NormalTok{(}\DataTypeTok{x =} \StringTok{"Year"}\NormalTok{, }\DataTypeTok{y =} \StringTok{"Bear Abundance"}\NormalTok{, }\DataTypeTok{title =} \StringTok{"Yearly Grizzly Bear population 1978 - 1997"}\NormalTok{)}\OperatorTok{+}
\StringTok{  }\KeywordTok{theme}\NormalTok{(}\DataTypeTok{text =} \KeywordTok{element_text}\NormalTok{(}\DataTypeTok{family =} \StringTok{"Times"}\NormalTok{))}

\NormalTok{grizzly_plot_}\DecValTok{3}
\end{Highlighting}
\end{Shaded}

\includegraphics{answers_files/figure-latex/unnamed-chunk-5-1.pdf}

\begin{Shaded}
\begin{Highlighting}[]
\CommentTok{#statistics - how has the population declined?}

\NormalTok{late_diff <-}\StringTok{ }\NormalTok{(grizzly_late}\OperatorTok{$}\NormalTok{N[}\DecValTok{10}\NormalTok{]}\OperatorTok{-}\NormalTok{grizzly_late}\OperatorTok{$}\NormalTok{N[}\DecValTok{1}\NormalTok{])}

\NormalTok{late_pyr <-}\StringTok{ }\NormalTok{late_diff}\OperatorTok{/}\DecValTok{10}

\CommentTok{#the total percent decline}

\NormalTok{late_perc <-}\StringTok{ }\NormalTok{late_diff}\OperatorTok{/}\KeywordTok{sum}\NormalTok{(grizzly_early}\OperatorTok{$}\NormalTok{N)}

\CommentTok{#use linear regression to compare slopes - is the slope the rate of decline?}

\NormalTok{late_model <-}\KeywordTok{lm}\NormalTok{(grizzly_late}\OperatorTok{$}\NormalTok{N }\OperatorTok{~}\StringTok{ }\NormalTok{grizzly_late}\OperatorTok{$}\NormalTok{Year)}

\KeywordTok{summary}\NormalTok{(late_model)}
\end{Highlighting}
\end{Shaded}

For the last 10 years represented in the data set, there appears to be
an increase in bear population, with 22 more bears in 1979 than in 1968.
Using linear regression, this amounts to an on average incease of 3.15
bears per year.

\textbf{9. Write a short paragraph to the park superintendent describing
your conclusions on the effects (if any) of the dump closures on the
grizzly bear population. (1 point)}

While the dump closures appeared to coincide with a decline in the
grizzly bear population, our analysis suggests that it actually had very
little short term effect. The bear popluation was declining prior to the
dump closures at about -.76 bears per year, and this rate remained
approximately the same for the 10 year period after. Longer term, the
dump closures may have had a positive effect on the grizzly population,
as bear abundance increased between 1979 and 1997 at an average rate of
approximately 3 bears per year. This may suggest that while dump
closures may have at first increased human bear conflict, in the long
term the decision helped keep bears away from the more densely populated
areas of the park and decreased conflicts.

\end{document}
